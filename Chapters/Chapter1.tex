
% Chapter Template

\chapter{Project Focus}\label{chapter:firstchapter} % Main chapter title

\label{ChapterX} % Change X to a consecutive number; for referencing this chapter elsewhere, use \ref{ChapterX}

%----------------------------------------------------------------------------------------
%	SECTION 1
%----------------------------------------------------------------------------------------

\section{Problem}\label{sec:firstsection}

% It is a good idea to have each sentence on a separate line, so that if you get feedback or changes from someone else
% the diffs will be much easier to manage
The problems occur while interfacing to human can cause error with the irregular signal acquiring when the irregular heart beats by the various movements of the patient, artifacts also causes the irregular detection.
Under these various conditions the oximeter deliver irregular results.

a) Carbon Monoxide: As the carbon monoxide is in small amount that direclty interacts to the haemoglobin that which shows the saturation reading variation for an example if heamoglobin consists of 15 percent of heamoglobin and 80 percent of oxygen then the final result display as 95 percent and in these case these cannot be used for the people who is a chain smoker.

b) Anemia: Low quantity of heamoglobin causes heamoglobin defficiency normally it ranges to 11-18g/dl.

c) External interference: Inaccurate readings may occur while taking measurementwhen it is exposed to strong light.

d) The pulse oximeter cannot detect the accurate measurements whether the nail is polished or pressed.


%\begin{figure}
%\begin{centering}
%\includegraphics[width=10cm,height=10cm,keepaspectratio]{Figures/dont-panic-e1534046233310.jpg}
%\caption{The Hitch Hiker's Guide To The Galaxy (not to be confused with \cite{Reference1}. Image Credit David Strine (License: CC0)}
%\label{fig:ThisFig}
%\end{centering}
%\end{figure}

%-----------------------------------
%	SUBSECTION 1
%-----------------------------------
%\subsection{Subsection 1}



%-----------------------------------
%	SUBSECTION 2
%-----------------------------------

%\subsection{Subsection 2}

%----------------------------------------------------------------------------------------
%	SECTION 2
%----------------------------------------------------------------------------------------

\section{Research Questions}

1. The Pulse Oximetry uses minimum two wavelengths because to improve its sensitivity and it distinguishes the how much light is absorbed by non pulsatile tissue with the constant/DC and the light obsorbed by the arterial blood with non constant/AC in which the both will be analysed by the meter with the help of the microprocessor with 660nm and 940nm by IR ratio which absorbs by the oxyhaemoglobin and deoxyhaemoglobin.

2. PPG filterisation with the least delay with high noise cancellation with the local maxima and local minima of the DC component offset can be given as 
\begin{center}
R=In(Rmin/Rmax)/(IRmin/IRmax)
\end{center}
where Rmin and Rmax denotes the red light intensity and the IRmin and IRmax denotes the infrared light intensity.

3. Calibration of Pulse oximeter system is done by (S= 110-25*R). The ratio of the R is caliculated by ac and dc absorbance as R=(ac absorbance/dc absorbance)red/(ac absorbance/dc absorbance)IR.

Where ac= pulsatile arterial blood and dc= tissue + capillaryblood + venuous blood + non pulsatile arterial blood.


%----------------------------------------------------------------------------------------
%	SECTION 3
%----------------------------------------------------------------------------------------

\section{Background Survey}

Survey 1:- An overview on heart rate monitoring and pulse oximeter system.

				For the signal processing the algorithm in the real time for the pulse oximeter PIC18F452 is used to find the intensity and the timing for IR and Red light. From the light with the help of DAC converter is helpful to derive the value of SpO2 and also calculates the number of beats per minute. For the data transferring it is connected to PC USART.
				

Survey 2 :- Reflectance based pulse oximeter for the chest and wrist.

				Signal processing were done by sampled using LabVIEW by DAQ assistant with the analog inputs by NI ELVIS board with 100Hz sampling frequency.
					
					
Survey 3:- Design and evaluation of a low cost smartphone pulse oximeter.

				The finger tip pulse oximeter has interfaced with mobile application through a signal processing unit that converts the raw probe signals into realtime audio signals. The Aux cable connected to the mobile phone sends the data through and mobile application in the phone which converts the audio data to readable outputs. 

