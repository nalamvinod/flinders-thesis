%%%%%%%%%%%%%%%%%%%%%%%%%%%%%%%%%%%%%%%%%
% Masters/Doctoral Thesis 
% LaTeX Template
% Version 2.5 (27/8/17)
%
% This template was downloaded from:
% http://www.LaTeXTemplates.com
%
% Version 2.x major modifications by:
% Vel (vel@latextemplates.com)
%
% This template is based on a template by:
% Steve Gunn (http://users.ecs.soton.ac.uk/srg/softwaretools/document/templates/)
% Sunil Patel (http://www.sunilpatel.co.uk/thesis-template/)
%
% Template license:
% CC BY-NC-SA 3.0 (http://creativecommons.org/licenses/by-nc-sa/3.0/)
%
%%%%%%%%%%%%%%%%%%%%%%%%%%%%%%%%%%%%%%%%%

%----------------------------------------------------------------------------------------
%	PACKAGES AND OTHER DOCUMENT CONFIGURATIONS
%----------------------------------------------------------------------------------------

\documentclass[
12pt, % The default document font size, options: 10pt, 11pt, 12pt
oneside, % Two side (alternating margins) for binding by default, uncomment to switch to one side
english, % ngerman for German
onehalfspacing, % Single line spacing, alternatives: onehalfspacing or doublespacing
%draft, % Uncomment to enable draft mode (no pictures, no links, overfull hboxes indicated)
%nolistspacing, % If the document is onehalfspacing or doublespacing, uncomment this to set spacing in lists to single
%liststotoc, % Uncomment to add the list of figures/tables/etc to the table of contents
%toctotoc, % Uncomment to add the main table of contents to the table of contents
%parskip, % Uncomment to add space between paragraphs
%nohyperref, % Uncomment to not load the hyperref package
headsepline, % Uncomment to get a line under the header
%chapterinoneline, % Uncomment to place the chapter title next to the number on one line
%consistentlayout, % Uncomment to change the layout of the declaration, abstract and acknowledgements pages to match the default layout
]{MastersDoctoralThesis} % The class file specifying the document structure

\usepackage[utf8]{inputenc} % Required for inputting international characters
\usepackage[T1]{fontenc} % Output font encoding for international characters
\usepackage{todonotes}
\usepackage{mathpazo} % Use the Palatino font by default

%\usepackage[style=numeric]{biblatex} % Use the bibtex backend with the authoryear citation style (which resembles APA)
\usepackage[backend=bibtex,style=numeric]{biblatex}
\bibliography{main} 

\usepackage[autostyle=true]{csquotes} % Required to generate language-dependent quotes in the bibliography

%----------------------------------------------------------------------------------------
%	MARGIN SETTINGS
%----------------------------------------------------------------------------------------

\geometry{
	paper=a4paper, % Change to letterpaper for US letter
	inner=2.5cm, % Inner margin
	outer=3.8cm, % Outer margin
	bindingoffset=.5cm, % Binding offset
	top=1.5cm, % Top margin
	bottom=1.5cm, % Bottom margin
	%showframe, % Uncomment to show how the type block is set on the page
}

%----------------------------------------------------------------------------------------
%	THESIS INFORMATION
%----------------------------------------------------------------------------------------

\thesistitle{Low cost Biomedical device platform with the parameter SpO2} % Your thesis title, this is used in the title and abstract, print it elsewhere with \ttitle
\supervisor{Dr. Paul \textsc{Gardner-Stephen}} % Your supervisor's name, this is used in the title page, print it elsewhere with \supname
\examiner{} % Your examiner's name, this is not currently used anywhere in the template, print it elsewhere with \examname
\degree{Master of Engineering(Biomedical)} % Your degree name, this is used in the title page and abstract, print it elsewhere with \degreename
\author{Teja Sai vinod Nalam} % Your name, this is used in the title page and abstract, print it elsewhere with \authorname
\addresses{} % Your address, this is not currently used anywhere in the template, print it elsewhere with \addressname

\subject{Biomedical Engineering} % Your subject area, this is not currently used anywhere in the template, print it elsewhere with \subjectname
\keywords{} % Keywords for your thesis, this is not currently used anywhere in the template, print it elsewhere with \keywordnames
\university{\href{https://www.flinders.edu.au/}{Flinders University}} % Your university's name and URL, this is used in the title page and abstract, print it elsewhere with \univname
\department{\href{https://www.flinders.edu.au/college-science-engineering}{College of Science and Engineering}} % Your department's name and URL, this is used in the title page and abstract, print it elsewhere with \deptname
\group{The Douglas Adams Institute For Implausible Linguistics} % Your research group's name and URL, this is used in the title page, print it elsewhere with \groupname
\faculty{\href{}{}} % Your faculty's name and URL, this is used in the title page and abstract, print it elsewhere with \facname

\AtBeginDocument{
\hypersetup{pdftitle=\ttitle} % Set the PDF's title to your title
\hypersetup{pdfauthor=\authorname} % Set the PDF's author to your name
\hypersetup{pdfkeywords=\keywordnames} % Set the PDF's keywords to your keywords
}

\begin{document}

\frontmatter % Use roman page numbering style (i, ii, iii, iv...) for the pre-content pages

\pagestyle{plain} % Default to the plain heading style until the thesis style is called for the body content

%----------------------------------------------------------------------------------------
%	TITLE PAGE
%----------------------------------------------------------------------------------------

\begin{titlepage}
\begin{center}

\vspace*{.06\textheight}
{\scshape\LARGE \univname\par}\vspace{1.5cm} % University name
\textsc{\Large Masters Project}\\[0.5cm] % Thesis type

\HRule \\[0.4cm] % Horizontal line
{\huge \bfseries \ttitle\par}\vspace{0.4cm} % Thesis title
\HRule \\[1.5cm] % Horizontal line
 
\begin{minipage}[t]{0.4\textwidth}
\begin{flushleft} \large
\emph{Author:}\\
{\authorname} % Author name - remove the \href bracket to remove the link
\end{flushleft}
\end{minipage}
\begin{minipage}[t]{0.4\textwidth}
\begin{flushright} \large
\emph{Supervisor:} \\
{\supname} % Supervisor name - remove the \href bracket to remove the link  
\end{flushright}
\end{minipage}\\[3cm]
 
\vfill

\large \textit{A thesis submitted in fulfilment of the requirements\\ Master of Engineering(Biomedical) \degreename}\\[0.3cm] % University requirement text
%%\groupname\\\deptname\\[2cm] % Research group name and department name
 
\vfill

{\large \today}\\[4cm] % Date
%\includegraphics{Logo} % University/department logo - uncomment to place it
 
\vfill
\end{center}
\end{titlepage}

%----------------------------------------------------------------------------------------
%	DECLARATION PAGE
%----------------------------------------------------------------------------------------

\begin{declaration}
\addchaptertocentry{\authorshipname} % Add the declaration to the table of contents
\noindent I, \authorname, declare that this thesis titled, \enquote{\ttitle} and the work presented in it are my own. I confirm that:

\begin{itemize} 
\item This work was done wholly while in candidature for a degree of \degreename.
\item This document is in accordance with the plagiarism policy of \univname.
\item Where any part of this thesis has previously been submitted for a degree or any other qualification at this University or any other institution, this has been clearly stated.
\item Where I have consulted the published work of others, this is always clearly attributed.
\item Where I have quoted from the work of others, the source is always given. With the exception of such quotations, this thesis is entirely my own work.
\item I have acknowledged all main sources of help.
\item Where the thesis is based on work done by myself jointly with others, I have made clear exactly what was done by others and what I have contributed myself.\\
\end{itemize}
 
\noindent Signed:\\
\rule[0.5em]{25em}{0.5pt} % This prints a line for the signature
 
\noindent Date:\\
\rule[0.5em]{25em}{0.5pt} % This prints a line to write the date
\end{declaration}

\cleardoublepage

%----------------------------------------------------------------------------------------
%	QUOTATION PAGE
%----------------------------------------------------------------------------------------

\vspace*{0.2\textheight}

\noindent\enquote{\itshape One of the major problem in the modern world to the human is running behind the money and neglects the monitoring of his own health. The health monitoring with several parameters which includes the remote monitoring and self management with an innovation, the companies are gaining more credentials in the name of fame and huge profits in international market and which replacing the health care centers.

This thesis discuss the continuous monitoring with the human interface in unique way known as human-monitoring-physician-management, whose having a large scope for the bioengineers, architects, designers. This monitoring can also be called as Smart handy human interface monitoring system for not only one person and also for whole family world in a low cost platform.

SpO2 is also known as the Pulse Oximetry is a non invasive peripheral device in a carryiable mode device which monitors the oxygen saturation of the human which depends upon the arterial blood flow which excludes venous blood in other words it measures the hemoglobin saturation. As per the iData research in US, the pulse Oximetry monitoring with 700 million USD in 2011 for the equipment and the sensors. }\bigbreak

\hfill 

%----------------------------------------------------------------------------------------
%	ABSTRACT PAGE
%----------------------------------------------------------------------------------------

\begin{abstract}
  \addchaptertocentry{\abstractname} % Add the abstract to the table of contents

One of the major problem in the modern world to the human is running behind the money and neglects the monitoring of his own health. The health monitoring with several parameters which includes the remote monitoring and self management with an innovation, the companies are gaining more credentials in the name of fame and huge profits in international market and which replacing the health care centers.

This thesis discuss the continuous monitoring with the human interface in unique way known as human-monitoring-physician-management, whose having a large scope for the bioengineers, architects, designers. This monitoring can also be called as Smart handy human interface monitoring system for not only one person and also for whole family world in a low cost platform.

SpO2 is also known as the Pulse Oximetry is a non invasive peripheral device in a carryiable mode device which monitors the oxygen saturation of the human which depends upon the arterial blood flow which excludes venous blood in other words it measures the hemoglobin saturation. As per the iData research in US, the pulse Oximetry monitoring with 700 million USD in 2011 for the equipment and the sensors.

\end{abstract}

%----------------------------------------------------------------------------------------
%	ACKNOWLEDGEMENTS
%----------------------------------------------------------------------------------------

\begin{acknowledgements}
\addchaptertocentry{\acknowledgementname} % Add the acknowledgements to the table of contents
I wish to thank the editorial staff of Megadodo Publications (Ursa Minor Beta) for their support throughout this thesis,
as well as Clearance Textiles Limited of Strood for providing the numerous bath towels consumed as part of this thesis.
\end{acknowledgements}

%----------------------------------------------------------------------------------------
%	LIST OF CONTENTS/FIGURES/TABLES PAGES
%----------------------------------------------------------------------------------------

\tableofcontents % Prints the main table of contents

\listoffigures % Prints the list of figures

%\listoftables % Prints the list of tables

%----------------------------------------------------------------------------------------
%	ABBREVIATIONS
%----------------------------------------------------------------------------------------

%\begin{abbreviations}{ll} % Include a list of abbreviations (a table of two columns)

%\textbf{LAH} & \textbf{L}ist \textbf{A}bbreviations \textbf{H}ere\\
%\textbf{WSF} & \textbf{W}hat (it) \textbf{S}tands \textbf{F}or\\

%\end{abbreviations}

%----------------------------------------------------------------------------------------
%	PHYSICAL CONSTANTS/OTHER DEFINITIONS
%----------------------------------------------------------------------------------------

%\begin{constants}{lr@{${}={}$}l} % The list of physical constants is a three column table

% The \SI{}{} command is provided by the siunitx package, see its documentation for instructions on how to use it

%Speed of Light & $c_{0}$ & \SI{2.99792458e8}{\meter\per\second} (exact)\\
%Constant Name & $Symbol$ & $Constant Value$ with units\\

%\end{constants}

%----------------------------------------------------------------------------------------
%	SYMBOLS
%----------------------------------------------------------------------------------------

%\begin{symbols}{lll} % Include a list of Symbols (a three column table)

%$a$ & distance & \si{\meter} \\
%$P$ & power & \si{\watt} (\si{\joule\per\second}) \\
%Symbol & Name & Unit \\

%\addlinespace % Gap to separate the Roman symbols from the Greek

%$\omega$ & angular frequency & \si{\radian} \\

%\end{symbols}

%----------------------------------------------------------------------------------------
%	DEDICATION
%----------------------------------------------------------------------------------------

%\dedicatory{For/Dedicated to/To my\ldots} 

%----------------------------------------------------------------------------------------
%	THESIS CONTENT - CHAPTERS
%----------------------------------------------------------------------------------------

\mainmatter % Begin numeric (1,2,3...) page numbering

\pagestyle{thesis} % Return the page headers back to the "thesis" style

% Include the chapters of the thesis as separate files from the Chapters folder
% Uncomment the lines as you write the chapters


% Chapter Template

\chapter{Chapter Title Here}\label{chapter:firstchapter} % Main chapter title

\label{ChapterX} % Change X to a consecutive number; for referencing this chapter elsewhere, use \ref{ChapterX}

%----------------------------------------------------------------------------------------
%	SECTION 1
%----------------------------------------------------------------------------------------

\section{Main Section 1}\label{sec:firstsection}

% It is a good idea to have each sentence on a separate line, so that if you get feedback or changes from someone else
% the diffs will be much easier to manage
Lorem ipsum dolor (Figure \ref{fig:ThisFig}) sit amet, consectetur adipiscing elit\cite{fleischman1994pragmatics}.
Aliquam ultricies lacinia euismod.
Nam tempus risus in dolor rhoncus in interdum enim tincidunt.
Donec vel nunc neque.
In condimentum ullamcorper quam non consequat.
Fusce sagittis tempor feugiat.
Fusce magna erat, molestie eu convallis ut, tempus sed arcu.
Quisque molestie, ante a tincidunt ullamcorper, sapien enim dignissim lacus, in semper nibh erat lobortis purus.
Integer dapibus ligula ac risus convallis pellentesque.

\begin{figure}
\begin{centering}
\includegraphics[width=10cm,height=10cm,keepaspectratio]{Figures/dont-panic-e1534046233310.jpg}
\caption{The Hitch Hiker's Guide To The Galaxy (not to be confused with \cite{Reference1}. Image Credit David Strine (License: CC0)}
\label{fig:ThisFig}
\end{centering}
\end{figure}

%-----------------------------------
%	SUBSECTION 1
%-----------------------------------
\subsection{Subsection 1}

Nunc posuere quam at lectus tristique eu ultrices augue venenatis (Chapter \ref{chapter:firstchapter}).
Vestibulum ante ipsum primis in faucibus orci luctus et ultrices posuere cubilia Curae; Aliquam erat volutpat.
Vivamus sodales tortor eget quam adipiscing in vulputate ante ullamcorper.
Sed eros ante, lacinia et sollicitudin et, aliquam sit amet augue.
In hac habitasse platea dictumst (Section \ref{sec:firstsection}).

%-----------------------------------
%	SUBSECTION 2
%-----------------------------------

\subsection{Subsection 2}
Morbi rutrum odio eget arcu adipiscing sodales.
Aenean et purus a est pulvinar pellentesque.
 Cras in elit neque, quis varius elit.
 Phasellus fringilla, nibh eu tempus venenatis, dolor elit posuere quam, quis adipiscing urna leo nec orci.
 Sed nec nulla auctor odio aliquet consequat.
 Ut nec nulla in ante ullamcorper aliquam at sed dolor.
 Phasellus fermentum magna in augue gravida cursus.
 Cras sed pretium lorem.
 Pellentesque eget ornare odio.
 Proin accumsan, massa viverra cursus pharetra, ipsum nisi lobortis velit, a malesuada dolor lorem eu neque.

%----------------------------------------------------------------------------------------
%	SECTION 2
%----------------------------------------------------------------------------------------

\section{Main Section 2}

Sed ullamcorper quam eu nisl interdum at interdum enim egestas.
 Aliquam placerat justo sed lectus lobortis ut porta nisl porttitor.
 Vestibulum mi dolor, lacinia molestie gravida at, tempus vitae ligula.
 Donec eget quam sapien, in viverra eros.
 Donec pellentesque justo a massa fringilla non vestibulum metus vestibulum.
 Vestibulum in orci quis felis tempor lacinia.
 Vivamus ornare ultrices facilisis.
 Ut hendrerit volutpat vulputate.
 Morbi condimentum venenatis augue, id porta ipsum vulputate in.
 Curabitur luctus tempus justo.
 Vestibulum risus lectus, adipiscing nec condimentum quis, condimentum nec nisl.
 Aliquam dictum sagittis velit sed iaculis.
 Morbi tristique augue sit amet nulla pulvinar id facilisis ligula mollis.
 Nam elit libero, tincidunt ut aliquam at, molestie in quam.
 Aenean rhoncus vehicula hendrerit.

%\include{Chapters/Chapter2} 
%\include{Chapters/Chapter3}
%\include{Chapters/Chapter4} 
%\include{Chapters/Chapter5} 
%\include{Chapters/Chapter6}
%\include{Chapters/Chapter7}
%\include{Chapters/Chapter8}
%\include{Chapters/FavouriteChapter}

%----------------------------------------------------------------------------------------
%	THESIS CONTENT - APPENDICES
%----------------------------------------------------------------------------------------

\appendix % Cue to tell LaTeX that the following "chapters" are Appendices

% Include the appendices of the thesis as separate files from the Appendices folder
% Uncomment the lines as you write the Appendices

\include{Appendices/AppendixA}
%\include{Appendices/AppendixB}
%\include{Appendices/AppendixC}

%----------------------------------------------------------------------------------------
%	BIBLIOGRAPHY
%----------------------------------------------------------------------------------------

\printbibliography

\end{document}  
